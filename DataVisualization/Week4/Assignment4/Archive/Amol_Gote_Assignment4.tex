\documentclass[]{article}
\usepackage{lmodern}
\usepackage{amssymb,amsmath}
\usepackage{ifxetex,ifluatex}
\usepackage{fixltx2e} % provides \textsubscript
\ifnum 0\ifxetex 1\fi\ifluatex 1\fi=0 % if pdftex
  \usepackage[T1]{fontenc}
  \usepackage[utf8]{inputenc}
\else % if luatex or xelatex
  \ifxetex
    \usepackage{mathspec}
  \else
    \usepackage{fontspec}
  \fi
  \defaultfontfeatures{Ligatures=TeX,Scale=MatchLowercase}
\fi
% use upquote if available, for straight quotes in verbatim environments
\IfFileExists{upquote.sty}{\usepackage{upquote}}{}
% use microtype if available
\IfFileExists{microtype.sty}{%
\usepackage{microtype}
\UseMicrotypeSet[protrusion]{basicmath} % disable protrusion for tt fonts
}{}
\usepackage[margin=1in]{geometry}
\usepackage{hyperref}
\hypersetup{unicode=true,
            pdftitle={Assignment 04},
            pdfauthor={Amol Gote},
            pdfborder={0 0 0},
            breaklinks=true}
\urlstyle{same}  % don't use monospace font for urls
\usepackage{color}
\usepackage{fancyvrb}
\newcommand{\VerbBar}{|}
\newcommand{\VERB}{\Verb[commandchars=\\\{\}]}
\DefineVerbatimEnvironment{Highlighting}{Verbatim}{commandchars=\\\{\}}
% Add ',fontsize=\small' for more characters per line
\usepackage{framed}
\definecolor{shadecolor}{RGB}{248,248,248}
\newenvironment{Shaded}{\begin{snugshade}}{\end{snugshade}}
\newcommand{\AlertTok}[1]{\textcolor[rgb]{0.94,0.16,0.16}{#1}}
\newcommand{\AnnotationTok}[1]{\textcolor[rgb]{0.56,0.35,0.01}{\textbf{\textit{#1}}}}
\newcommand{\AttributeTok}[1]{\textcolor[rgb]{0.77,0.63,0.00}{#1}}
\newcommand{\BaseNTok}[1]{\textcolor[rgb]{0.00,0.00,0.81}{#1}}
\newcommand{\BuiltInTok}[1]{#1}
\newcommand{\CharTok}[1]{\textcolor[rgb]{0.31,0.60,0.02}{#1}}
\newcommand{\CommentTok}[1]{\textcolor[rgb]{0.56,0.35,0.01}{\textit{#1}}}
\newcommand{\CommentVarTok}[1]{\textcolor[rgb]{0.56,0.35,0.01}{\textbf{\textit{#1}}}}
\newcommand{\ConstantTok}[1]{\textcolor[rgb]{0.00,0.00,0.00}{#1}}
\newcommand{\ControlFlowTok}[1]{\textcolor[rgb]{0.13,0.29,0.53}{\textbf{#1}}}
\newcommand{\DataTypeTok}[1]{\textcolor[rgb]{0.13,0.29,0.53}{#1}}
\newcommand{\DecValTok}[1]{\textcolor[rgb]{0.00,0.00,0.81}{#1}}
\newcommand{\DocumentationTok}[1]{\textcolor[rgb]{0.56,0.35,0.01}{\textbf{\textit{#1}}}}
\newcommand{\ErrorTok}[1]{\textcolor[rgb]{0.64,0.00,0.00}{\textbf{#1}}}
\newcommand{\ExtensionTok}[1]{#1}
\newcommand{\FloatTok}[1]{\textcolor[rgb]{0.00,0.00,0.81}{#1}}
\newcommand{\FunctionTok}[1]{\textcolor[rgb]{0.00,0.00,0.00}{#1}}
\newcommand{\ImportTok}[1]{#1}
\newcommand{\InformationTok}[1]{\textcolor[rgb]{0.56,0.35,0.01}{\textbf{\textit{#1}}}}
\newcommand{\KeywordTok}[1]{\textcolor[rgb]{0.13,0.29,0.53}{\textbf{#1}}}
\newcommand{\NormalTok}[1]{#1}
\newcommand{\OperatorTok}[1]{\textcolor[rgb]{0.81,0.36,0.00}{\textbf{#1}}}
\newcommand{\OtherTok}[1]{\textcolor[rgb]{0.56,0.35,0.01}{#1}}
\newcommand{\PreprocessorTok}[1]{\textcolor[rgb]{0.56,0.35,0.01}{\textit{#1}}}
\newcommand{\RegionMarkerTok}[1]{#1}
\newcommand{\SpecialCharTok}[1]{\textcolor[rgb]{0.00,0.00,0.00}{#1}}
\newcommand{\SpecialStringTok}[1]{\textcolor[rgb]{0.31,0.60,0.02}{#1}}
\newcommand{\StringTok}[1]{\textcolor[rgb]{0.31,0.60,0.02}{#1}}
\newcommand{\VariableTok}[1]{\textcolor[rgb]{0.00,0.00,0.00}{#1}}
\newcommand{\VerbatimStringTok}[1]{\textcolor[rgb]{0.31,0.60,0.02}{#1}}
\newcommand{\WarningTok}[1]{\textcolor[rgb]{0.56,0.35,0.01}{\textbf{\textit{#1}}}}
\usepackage{graphicx,grffile}
\makeatletter
\def\maxwidth{\ifdim\Gin@nat@width>\linewidth\linewidth\else\Gin@nat@width\fi}
\def\maxheight{\ifdim\Gin@nat@height>\textheight\textheight\else\Gin@nat@height\fi}
\makeatother
% Scale images if necessary, so that they will not overflow the page
% margins by default, and it is still possible to overwrite the defaults
% using explicit options in \includegraphics[width, height, ...]{}
\setkeys{Gin}{width=\maxwidth,height=\maxheight,keepaspectratio}
\IfFileExists{parskip.sty}{%
\usepackage{parskip}
}{% else
\setlength{\parindent}{0pt}
\setlength{\parskip}{6pt plus 2pt minus 1pt}
}
\setlength{\emergencystretch}{3em}  % prevent overfull lines
\providecommand{\tightlist}{%
  \setlength{\itemsep}{0pt}\setlength{\parskip}{0pt}}
\setcounter{secnumdepth}{0}
% Redefines (sub)paragraphs to behave more like sections
\ifx\paragraph\undefined\else
\let\oldparagraph\paragraph
\renewcommand{\paragraph}[1]{\oldparagraph{#1}\mbox{}}
\fi
\ifx\subparagraph\undefined\else
\let\oldsubparagraph\subparagraph
\renewcommand{\subparagraph}[1]{\oldsubparagraph{#1}\mbox{}}
\fi

%%% Use protect on footnotes to avoid problems with footnotes in titles
\let\rmarkdownfootnote\footnote%
\def\footnote{\protect\rmarkdownfootnote}

%%% Change title format to be more compact
\usepackage{titling}

% Create subtitle command for use in maketitle
\providecommand{\subtitle}[1]{
  \posttitle{
    \begin{center}\large#1\end{center}
    }
}

\setlength{\droptitle}{-2em}

  \title{Assignment 04}
    \pretitle{\vspace{\droptitle}\centering\huge}
  \posttitle{\par}
    \author{Amol Gote}
    \preauthor{\centering\large\emph}
  \postauthor{\par}
      \predate{\centering\large\emph}
  \postdate{\par}
    \date{2/10/2020}


\begin{document}
\maketitle

\hypertarget{question-1}{%
\section{Question 1}\label{question-1}}

What is happening to price over time (yr\_built)

\begin{Shaded}
\begin{Highlighting}[]
\NormalTok{houses <-}\StringTok{ }\KeywordTok{read_csv}\NormalTok{(}\StringTok{"data/KING COUNTY House Data.csv"}\NormalTok{)}
\end{Highlighting}
\end{Shaded}

\begin{verbatim}
## Parsed with column specification:
## cols(
##   .default = col_double(),
##   date = col_datetime(format = "")
## )
\end{verbatim}

\begin{verbatim}
## See spec(...) for full column specifications.
\end{verbatim}

\begin{Shaded}
\begin{Highlighting}[]
\CommentTok{#averagePriceEachYear <- houses %>%}
\CommentTok{#  group_by(yr_built) %>%}
\CommentTok{#  summarise(averagePrice = mean(price), medianPrice = median(price))}

\KeywordTok{ggplot}\NormalTok{(}\DataTypeTok{data =}\NormalTok{ houses, }\KeywordTok{aes}\NormalTok{(}\DataTypeTok{x =}\NormalTok{ yr_built, }\DataTypeTok{y =}\NormalTok{ price)) }\OperatorTok{+}
\StringTok{  }\KeywordTok{stat_summary}\NormalTok{(}\DataTypeTok{fun.y=}\NormalTok{mean,}\DataTypeTok{size=}\DecValTok{1}\NormalTok{,}\DataTypeTok{geom=}\StringTok{'line'}\NormalTok{, }\KeywordTok{aes}\NormalTok{(}\DataTypeTok{colour=}\StringTok{"Mean"}\NormalTok{)) }\OperatorTok{+}
\StringTok{  }\KeywordTok{stat_summary}\NormalTok{(}\DataTypeTok{fun.y=}\NormalTok{median,}\DataTypeTok{size=}\DecValTok{1}\NormalTok{,}\DataTypeTok{geom=}\StringTok{'line'}\NormalTok{, }\KeywordTok{aes}\NormalTok{(}\DataTypeTok{colour=}\StringTok{"Median"}\NormalTok{)) }\OperatorTok{+}
\StringTok{  }\KeywordTok{theme_minimal}\NormalTok{() }\OperatorTok{+}
\StringTok{  }\KeywordTok{scale_y_continuous}\NormalTok{(}\DataTypeTok{labels =}\NormalTok{ dollar) }\OperatorTok{+}
\StringTok{  }\KeywordTok{labs}\NormalTok{(}\DataTypeTok{x =} \StringTok{"Year"}\NormalTok{,}
  \DataTypeTok{y =} \StringTok{"Price"}\NormalTok{,}
  \DataTypeTok{title =} \StringTok{"Average price Year on Year"}\NormalTok{,}
  \DataTypeTok{subtitle =} \StringTok{"Created for Data Visulization Class - Assignment 4 - Question 1"}\NormalTok{,}
  \DataTypeTok{caption =} \StringTok{"Source: Data Visualization assignment 4"}\NormalTok{)}
\end{Highlighting}
\end{Shaded}

\includegraphics{Amol_Gote_Assignment4_files/figure-latex/unnamed-chunk-1-1.pdf}

\begin{Shaded}
\begin{Highlighting}[]
\CommentTok{#ggplot(data = houses, aes(x = yr_built, y = price)) +}
\CommentTok{#  stat_summary(fun.y=mean,size=1,geom='line', aes(colour="Mean")) +}
\CommentTok{#  stat_summary(fun.y=median,size=1,geom='line', aes(colour="Median")) +}
\CommentTok{#  theme_minimal() +}
\CommentTok{#  scale_y_continuous(labels = dollar) +}
\CommentTok{#  labs(x = "Year",}
\CommentTok{#  y = "Price",}
\CommentTok{#  title = "Average price Year on Year",}
\CommentTok{#  subtitle = "Created for Data Visulization Class - Assignment 4 - Question 1",}
\CommentTok{#  caption = "Source: Data Visualization assignment 4")}
\end{Highlighting}
\end{Shaded}

\begin{enumerate}
\def\labelenumi{\arabic{enumi}.}
\tightlist
\item
  For Comparing price year on year have taken mean and median of house
  prices for each year built.\\
\item
  Mean and median seems to be following the same pattern. Average price
  had peaked in early 1900's, have been dropping then till 1950's. Prior
  to 1950 it had dipped to its minimum number, reason for the same could
  be World War 2.
\item
  Post World war 2 it has started rising gradually with a dip around
  1970.
\item
  Post 1970 it has gradually went up with peaking in 2000 and then again
  dropping back due to economic depression in 2009. Post that that
  avarage price has recovered have hit peak, the peak numbers post 2009
  are same that of peak number is early 1900's.
\end{enumerate}

\hypertarget{question-2}{%
\section{Question 2}\label{question-2}}

What is happening to price over geographic space (Can be lat / long,
zipcode, etc)

\begin{Shaded}
\begin{Highlighting}[]
\NormalTok{counties <-}\StringTok{ }\KeywordTok{st_as_sf}\NormalTok{(}\KeywordTok{map}\NormalTok{(}\StringTok{"county"}\NormalTok{, }\DataTypeTok{plot =} \OtherTok{FALSE}\NormalTok{, }\DataTypeTok{fill =} \OtherTok{TRUE}\NormalTok{))}
\NormalTok{counties_wa <-counties }\OperatorTok\StringTok{ }
\StringTok{  }\KeywordTok{filter}\NormalTok{(}\KeywordTok{str_detect}\NormalTok{(ID, }\StringTok{'washington,'}\NormalTok{))}

\NormalTok{counties_wa_king <-}\StringTok{ }\NormalTok{counties_wa }\OperatorTok\StringTok{ }
\StringTok{  }\KeywordTok{filter}\NormalTok{(}\KeywordTok{str_detect}\NormalTok{(ID, }\StringTok{"king"}\NormalTok{)) }

\NormalTok{counties_wa_king }\OperatorTok\StringTok{ }
\StringTok{  }\KeywordTok{ggplot}\NormalTok{() }\OperatorTok{+}\StringTok{ }
\StringTok{  }\KeywordTok{geom_sf}\NormalTok{() }\OperatorTok{+}
\StringTok{  }\KeywordTok{geom_point}\NormalTok{(}\DataTypeTok{data =}\NormalTok{ houses, }\KeywordTok{aes}\NormalTok{(}\DataTypeTok{x =}\NormalTok{ long, }\DataTypeTok{y =}\NormalTok{ lat, }\DataTypeTok{color =}\NormalTok{ price), }\DataTypeTok{alpha=} \FloatTok{.05}\NormalTok{) }\OperatorTok{+}
\StringTok{  }\KeywordTok{scale_colour_viridis_c}\NormalTok{(}\StringTok{"Sale Price"}\NormalTok{, }\DataTypeTok{labels =}\NormalTok{ dollar) }\OperatorTok{+}
\StringTok{  }\KeywordTok{theme_minimal}\NormalTok{() }\OperatorTok{+}\StringTok{ }
\StringTok{  }\KeywordTok{labs}\NormalTok{(}\DataTypeTok{x =} \StringTok{"Latitude"}\NormalTok{,}
  \DataTypeTok{y =} \StringTok{"Longitude"}\NormalTok{,}
  \DataTypeTok{title =} \StringTok{"King County house prices over geographic space"}\NormalTok{)}
\end{Highlighting}
\end{Shaded}

\includegraphics{Amol_Gote_Assignment4_files/figure-latex/unnamed-chunk-3-1.pdf}
\# Question 3 What is happening to price over time and space?

\begin{Shaded}
\begin{Highlighting}[]
\NormalTok{counties_wa_king }\OperatorTok\StringTok{ }
\StringTok{  }\KeywordTok{ggplot}\NormalTok{() }\OperatorTok{+}\StringTok{ }
\StringTok{  }\KeywordTok{geom_sf}\NormalTok{() }\OperatorTok{+}
\StringTok{  }\KeywordTok{geom_point}\NormalTok{(}\DataTypeTok{data =}\NormalTok{ houses, }\KeywordTok{aes}\NormalTok{(}\DataTypeTok{x =}\NormalTok{ long, }\DataTypeTok{y =}\NormalTok{ lat, }\DataTypeTok{color =}\NormalTok{ price ), }\DataTypeTok{alpha=} \FloatTok{.05}\NormalTok{) }\OperatorTok{+}
\StringTok{  }\KeywordTok{scale_colour_viridis_c}\NormalTok{(}\StringTok{"Sale Price"}\NormalTok{, }\DataTypeTok{labels =}\NormalTok{ dollar) }\OperatorTok{+}
\StringTok{  }\KeywordTok{facet_wrap}\NormalTok{(}\OperatorTok{~}\NormalTok{decade) }\OperatorTok{+}
\StringTok{  }\KeywordTok{theme}\NormalTok{(}\DataTypeTok{axis.text.x =} \KeywordTok{element_text}\NormalTok{(}\DataTypeTok{angle =}\DecValTok{50}\NormalTok{, }\DataTypeTok{hjust=}\FloatTok{0.75}\NormalTok{))}\OperatorTok{+}\StringTok{ }
\StringTok{  }\KeywordTok{labs}\NormalTok{(}\DataTypeTok{x =} \StringTok{"Latitude"}\NormalTok{,}
  \DataTypeTok{y =} \StringTok{"Longitude"}\NormalTok{,}
  \DataTypeTok{title =} \StringTok{"King County house prices over time and geographic space"}\NormalTok{)}
\end{Highlighting}
\end{Shaded}

\includegraphics{Amol_Gote_Assignment4_files/figure-latex/unnamed-chunk-4-1.pdf}

\hypertarget{question-3}{%
\section{Question 3}\label{question-3}}


\end{document}
